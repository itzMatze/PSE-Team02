\documentclass[a4paper]{scrreprt}
\setcounter{tocdepth}{3}
\setcounter{secnumdepth}{3}

\usepackage[german]{babel}
\usepackage[utf8]{inputenc}
\usepackage[T1]{fontenc}
\usepackage{ae}
\usepackage{graphicx}
\usepackage{lscape} % querformat
\usepackage{tabu}
\usepackage{hyperref}
\usepackage{glossaries}

\makeglossaries

\newglossaryentry{Token}
{
  name=Token,
  description={Komponente die zur Identifizierung und Authentifizierung von Benutzern und Gruppen dient}
}


\begin{document}
\title{Implementierungsbericht}
\author{Fangzhou Bian, Kathrin Blum, Matthias Bruns, \\Leonhard Duda, Tan Grumser, Yuguang Lin}
\maketitle
%\Footnote für Fußnoten
% Platzierung des Inhaltsverzeichnisses
\tableofcontents



\chapter{Einleitung}

Bei diesem Dokument handelt es sich um den Implementierungsbericht, welcher zeitgleich zur eigentlichen Implementierung verfasst wurde. Die Implementierungsphase knüpft an die Entwurfsphase an und es wird versucht, die in dieser Phase definierten Entwurfsziele umzusetzen. Da dies aber nicht immer geglückt ist, verweist dieses Dokument unter anderem auch auf Änderungen bzw. Abweichungen am Entwurf. Auf diese Änderungen wird im Folgenden noch genauer eingegangen. Des Weiteren wird darüber aufgeklärt, ob alle Pflichtkriterien implementiert wurden und ob sogar Wunschkriterien realisiert werden konnten. Es wird aber auch auf Probleme innerhalb der Implementierungsphase verwiesen und welche Funktionen besonders viel Zeit gekostet haben.

\chapter{Änderungen am Entwurf}




\printglossaries
\end{document}
