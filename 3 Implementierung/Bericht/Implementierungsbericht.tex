\documentclass[a4paper]{scrreprt}
\setcounter{tocdepth}{3}
\setcounter{secnumdepth}{3}

\usepackage[german]{babel}
\usepackage[utf8]{inputenc}
\usepackage[T1]{fontenc}
\usepackage{ae}
\usepackage{graphicx}
\usepackage{lscape} % querformat
\usepackage{tabu}
\usepackage{hyperref}
\usepackage{xcolor}
\usepackage[toc]{glossaries}

\makeglossaries

\newglossaryentry{MK10}
{
  name=MK10,
  description={Erstellen und Bearbeiten eines persönlichen Profils, bestehend aus: Einem eindeutigen Nutzernamen, Alter, Geschlecht, Motto,Profilbild (aus vorgegebener Auswahl), Status (Student, Professor, Auszubildende, Schüler, Sonstige)}
}

\newglossaryentry{MK20}
{
  name=MK20,
  description={Einsehen des Mensaplans des aktuellen Tages}
}

\newglossaryentry{MK30}
{
  name=MK30,
  description={Auswählen der Mensalinien/Mensawerke}
}

\newglossaryentry{MK40}
{
  name=MK40,
  description={Einsehen der Gruppen, mit übereinstimmenden Mensalinien/ -werken}
}

\newglossaryentry{MK50}
{
  name=MK50,
  description={Einstellen eines Zeitintervalls für das essen Gehen}
}

\newglossaryentry{MK60}
{
  name=MK60,
  description={Erstellen einer Gruppe}
}

\newglossaryentry{MK70}
{
  name=MK70,
  description={Beitreten einer Gruppe}
}

\newglossaryentry{MK80}
{
  name=MK80,
  description={Verlassen einer Gruppe}
}

\newglossaryentry{MK90}
{
  name=MK90,
  description={Anschauen der Profile der Gruppenmitglieder}
}

\newglossaryentry{MK100}
{
  name=MK100,
  description={Automatisches Löschen der Gruppe am Ende des Tages, oder wenn letztes Mitglied austritt}
}

\newglossaryentry{MK110}
{
  name=MK100,
  description={Administratorensicht: Administratoren sollen Gruppen und Nutzer löschen können}
}

\newglossaryentry{WK100}
{
  name=WK100,
  description={E-Mail Verifizierung für die Registrierung}
}



\begin{document}
\title{Implementierungsbericht}
\author{Fangzhou Bian, Kathrin Blum, Matthias Bruns, \\Leonhard Duda, Tan Grumser, Yuguang Lin}
\maketitle
%\Footnote für Fußnoten
% Platzierung des Inhaltsverzeichnisses
\tableofcontents



\chapter{Einleitung}

Bei diesem Dokument handelt es sich um den Implementierungsbericht, welcher zeitgleich zur eigentlichen Implementierung verfasst wurde. Die Implementierungsphase knüpft an die Entwurfsphase an und es wird versucht, die in dieser Phase definierten Entwurfsziele umzusetzen. Da dies aber nicht immer geglückt ist, verweist dieses Dokument unter anderem auch auf Änderungen bzw. Abweichungen am Entwurf. Auf diese Änderungen wird im Folgenden noch genauer eingegangen. Des Weiteren wird darüber aufgeklärt, ob alle Pflichtkriterien implementiert wurden und ob sogar Wunschkriterien realisiert werden konnten. Es wird aber auch auf Probleme innerhalb der Implementierungsphase verwiesen und welche Funktionen besonders viel Zeit gekostet haben.

\chapter{Änderungen am Entwurf}
\section{Server}
Ein grobe Änderung an der Form des Servers entstand dadurch, dass während des Entwurfs nicht klar war, dass der Server selbst keine Login- und Registrierungsfunktionen von der Firebase Admin SDK zur Verfügung gestellt bekommen wird. Dadurch wurden diese Funktionalitäten auf den Client ausgelagert und der Server verlor diese Funktionen. Es wurde in der View die AccountService Klasse gelöscht und bei den Controllern der AuthenticationController. Die Logik zum Anlegen von Usern wird jetzt vom UserController übernommen.
\subsection{Model}

\subsubsection{Änderung der internen Darstellung der Mensadaten}
Ursprünglich war geplant, die vom Studentenwerk bezogenen Mensadaten, also das tagesaktuelle Menü aller Linien, aufzuteilen. Also die Linien in ihre jeweiligen Mahlzeiten und die Mahlzeiten in ihren Namen, Preis, ihre Inhaltsstoffe und um welche Art Essen es sich dabei handelt. Diese Aufteilung wurde deshalb so gewählt, da man für zukünftige Funktionen geplant hat. Beispielweiße das Anzeigen eines Symbols als Indikator des Essenstyp(Normal, vegan, etc.). Diese Aufteilung wurde nun verworfen und die Mensadaten sind nun unterteilt in die Linien und diese Linien in ihre Mahlzeiten. Die Mahlzeiten werden also nicht mehr unterteilt. Die Begründung liegt darin, dass die bezogenen Daten inkonsistent sind. Beispielweiße verfügen manche Mahlzeiten über keinen Preis oder es fehlt die Angabe des Essenstyp. Außerdem werden nur die Preise für Studenten angezeigt, zu mindest bei der alternativen Ansicht, von der die Daten bezogen wurden.


\begin{itemize}
\item{Line.java}

\textcolor{red}{- enum Meal} \\
\textcolor{red}{- enum Ingredient} \\
\textcolor{red}{- enum FoodType} \\
\textcolor{green}{+ String name} \\
\textcolor{green}{+ String[] meals}\\
\end{itemize}

\subsubsection{Änderung der User Klasse}
Zusätzlich zu den bereits im Entwurfsdokument aufgezählten Attributen der User Klasse, wurde ein boolean hinzugefügt, der darüber aussagt ob ein User Administrationsrechte besitzt. Da diese Funktion bereits geplant war und auch als Pflichtkriterium aufgeführt wurde, handelt es sich hierbei nicht wirklich um eine Änderung. Viel mehr wurde einfach vergessen einen solchen boolean im Klassendiagramm aufzuführen.

\begin{itemize}
\item{User.java}

\textcolor{green}{+ boolean isAdmin}\\
\end{itemize}

\subsection{View}

\subsubsection{Änderung der User Service Klasse}
In der Klasse User Service wurde lediglich die Methode zum Erstellen eines Users hinzugefügt. Diese Änderung ist aufgrund der FireBase SDK notwendig, da unter anderem die Authentification Controller Klasse wegfällt und somit auch die Methode zum Registrieren. Die create User Methode ersetz demnach die register Methode.

\begin{itemize}
\item{UserService.java}

\textcolor{green}{+ void createUser(String token)}\\
\end{itemize}

\subsubsection{Account Verwaltung}
Durch neue Erkenntnisse bezüglich der FireBase SDK musste festgestellt werden, dass diese eine direkte Anmeldung an den Server aus Sicherheitsgründen nicht erlaubt. Aus diesem Grund kommuniziert der Client nun direkt mit FireBase, bekommt so durch seinen Token und stellt dann eine Anfrage an den Server. Aus diesem Grund fällt auch die Account Service Klasse komplett weg.

\begin{itemize}
\item\textcolor{red}{UserService.java}

\textcolor{red}{- login(Credentials credentials)}\\
\textcolor{red}{- register(Credentials credentials)}
\end{itemize}

\subsection{Controller}
\subsubsection{Änderung  des Mensa Data Controllers}
Die ursprüngliche Benennung dieser Klasse wurde verworfen und in MensaDataController geändert. Ebenso die Methoden getLineData() und updateLineData() zu getMensaData() und updateMensaData(). Wie bereits erwähnt, wurde die Klasse Line.java geändert und die einzelnen Gerichte sind nun einzelne Strings. Dies hat auch Auswirkungen auf diese Klasse, da die Strings an Speisen pro Linie nicht weiter aufgeteilt werden müssen. Um die Übersicht dieser Klasse zu gewährleisten, wurde außerdem eine Converter Methode hinzugefügt, die die abgefangenen Speisepläne in die gewünschte Form formatiert.

\begin{itemize}
\item{MensaDataController.java}

\textcolor{green}{+ MensaData Converter(String[] LineText)}\\
\end{itemize}

\subsubsection{Änderung des User Controllers}
Hier wurde eine Methode hinzugefügt, die sicherstellt, dass beim Hochfahren des Servers ein User mit Administrationsrechten erstellt wird. Mit diesem User können wir uns anmelden um eben die Funktionen eines Administrators auszuführen.

\begin{itemize}
\item{UserController.java}

\textcolor{green}{+ void initializeAdminUser()}\\
\end{itemize}

\subsubsection{Änderung des Group Controllers}
Hier wurden lediglich zwei Methoden hinzugefügt, die die Bedienbarkeit erleichtern. Eine Methode zum Löschen aller Gruppen und eine Methode die alle Gruppen zurückgibt.

\begin{itemize}
\item{GroupController.java}

\textcolor{green}{+ void removeAllGroups(}\\
\textcolor{green}{+ Group[] getAllGroups()}\\
\end{itemize}

\subsubsection{Alternative Authentifizierung}
Die Authentifizierung war in der Form in der sie definiert wurde aufgrund von FireBase nicht möglich. Deshalb wurde der Authentification Controller gestrichen. Zum Einsatz kommt nun die Klasse FirebaseAuthentifcator, welche sich nun um die Authentifizierung kümmert.

\begin{itemize}
\item\textcolor{red}{AuthentificationController.java}

\textcolor{red}{- User register(Credentials creds)}\\
\textcolor{red}{- User login(Credentials creds)}\\

\item\textcolor{green}{FirebaseAuthentifcator.java}

\textcolor{green}{+ String authenticateAndEncryptFirebaseTokenToUserToken(String firebaseToken)}\\
\textcolor{green}{+ void authenticateFirebaseToken(String firebaseToken)}
\end{itemize}

\subsubsection{Zusätzliche Repository Klassen}
Zur besseren Nutzung von Hibernate wurde zwei zusätzliche Repository Klasse realisiert. Das User Repository und das Group Repository

\begin{itemize}
\item\textcolor{green}{+ UserRepository.java}
\item\textcolor{green}{+ GroupRepository.java}
\end{itemize}


\section{Client}
\subsubsection{Die Konzepte des Entwurfs und ihre Umsetzung in der Implementierung}
Kernstück des Cliententwurfs ist die modulare Gliederung der Benutzeroberfläche. Dabei galt es, so viel sich Wiederholendes wie nur möglich in abstrahierten Oberklassen zusammenzufassen und die Klassenstruktur möglichst kompakt zu gestalten. Das Zentrum bilden dabei die Item-Klassen, die von MensaMeetItem abgeleitet werden. Diese sind jeweils für die Darstellung eines Elements zuständig, ob User, Group oder Line. Die jeweilige Klasse fasst dabei alle möglichen Darstellungsformen in sich zusammen: Sei es als kleines Listenelement (SMALL), als große, nicht editierbare Einzeldarstellung auf einer Seite (BIGNOTEDITABLE) und als editierbares Formular auf einer Seite (BIGEDITABLE). Dies beruht auf dem Gedanken, dass immer wieder dieselben Datensätze, nur jeweils in anderer Form dargestellt werden. 
Konkret wurde dies so umgesetzt, dass nacheinander alle Felder einer Klasse vom Namen über Zahlen bis hin zu Auswahllisten abgearbeitet wurden und jeweils für die drei Fälle gegebenenfalls unterschiedliche Darstellungen generiert wurden. Das Profilbild eines Users etwa wird in der Listenelement-Darstellung links vom Text dargestellt, in der Editierdarstellung dagegen als Auswahlkarussell über dem Text. Diese Fallunterscheidung kann jedoch auch in Methoden verlagert werden: so gibt es in der Item-Oberklasse die Methode createTextField, wobei mit TextField je nach Darstellungsform eine Textanzeige oder ein Eingabefeld gemeint sein kann, die jedoch beide dieselben Parameter empfangen. Was genommen wird, wird im Hintergrund entschieden, die Methode bleibt gleich. Dadurch ergibt sich eine übersichtlichere Darstellung. 
Eine weitere Bündelung von Informationen bestand darin, die Android-Ressourcen-IDs für Strings, die ein Element beschreiben, auch als IDs für diese Elemente genommen werden. Die Herausforderung bei der Implementierung bestand hauptsächlich darin, zu erkennen, welche Befehle gebündelt und verallgemeinert werden können und welche nicht. So musste ergänzend zum Entwurf eine weitere Klasse MensaMeetAdapter hinzugefügt werden, um die Listen und ihre Elemente, realisiert mit Reyclerviews, zusammenzuhalten. Diese Klasse wurde jedoch so gestaltet, dass sie nicht noch für jeden Datentyp (Linien, Gruppen, Benutzer) vererbt werden musste, sondern nur mit einem jeweiligen Typparameter zurechtkommt. Hierdurch hält sich die Anzahl der Klassen in Grenzen.
Android bietet die Möglichkeit an, Anzeigeelemente (Views) ineinander rekursiv zu verschachteln, sodass das modulare Konzept tatsächlich umgesetzt werden konnte. Allerdings wurde deutlich, dass Android nunmehr eher auf die Gestaltung von Benutzeroberflächen mit xml und grafischen Editoren setzt als auf ihre Generierung per Code. So lassen sich Layoutangaben wie Margin und Padding bequem in das XML schreiben, die programmatische Realisierung ist jedoch viel aufwändiger: Es muss müssen LayoutParams-Objekt abgeleitet werden, die jedoch nicht gleichzeitig Margin und Padding aufnehmen können. Außerdem sind die Folgen der Konfigurationen teilweise unberechenbar. Das hierbei verfolgte Ziel, mit so wenig Dateien wie möglich auszukommen und daher auf zusätzliche XML-Dateien zu verzichten, war daher nur sehr mühsam zu erreichen. 
Eine weitere Herausforderung war die Aufteilung in Darstellung (View) und Logik (Viewmodel). Nach dem Model-View-Viewmodel-Prinzip dürfen Viewmodels nicht direkt mit der View kommunizieren. Dies wird dadurch gelöst, dass die Views als Beobachter die Veränderung eines Statuswertes beim Viewmodel abonnieren. In der Activity SelectGroupsActivity, wo eine Gruppenliste mit verschachtelten Userlisten dargestellt wird und diese auch noch Buttons enthalten, die Befehle triggern, mussten alle Elemente von der Hauptactivity abonniert werden.

\chapter{Muss- und Wunschkriterien}
Die von uns definierten elf Musskriterien wurden alle vollständig implementiert. Das heißt, ein Nutzer kann sich sein persönliches Profil anlegen (\Gls{MK10}). Mit einem angelegten Profil kann er nun den tagesaktuellen Mensaplan einsehen(\Gls{MK20}). Hat er sich für eine Linie entschieden(\Gls{MK30}), so kann er nach dem festlegen eines Zeitintervalls(\Gls{MK50}) entweder einer vorgeschlagenen Gruppe beitreten(\Gls{MK40} \& \Gls{MK70}), oder selbst eine Gruppe erstellen(\Gls{MK60}). Egal ob er einer Gruppe beigetreten ist oder selbst eine erstellt hat, kann der User aus dieser Gruppe austreten(\Gls{MK80}). Innerhalb einer Gruppe kann er außerdem die Profile betrachten(\Gls{MK90}). Des Weiteren können Administratoren User und Gruppen löschen(\Gls{MK110} und Gruppen werden am Ende des Tages gelöscht(\Gls{MK100}).
Neben den Musskriterien wurde gezwungenermaßen auch das Wunschkriterium(\Gls{WK100}), also die E-Mail Verifizierung für die Registrierung, realisiert. Dies aufgrund von und durch FireBase. Die Abgrenzungskriterien wurden ebenfalls alle eingehalten.


\chapter{Unit tests}
\subsection{Server}

\subsubsection{Model}
Die Model-Klassen bedurften keinen Tests, da sie nur zur Datenhaltung dienen und lediglich aus gettern, settern und Konstruktoren bestehen.

\subsubsection{View}


\subsubsection{Controller}
\begin{itemize}
	\item \b getestete Funktionalitäten des Groupcontroller
	\begin{itemize}
		\item {Gültige Gruppe ins Repository hinzufügen: \\
		Eine Testgruppe wurde mit der Methode Group addGroup (Group group) ins Repository eingefügt. Die Methode gibt die gerade eingefügte Gruppe zurück.
Da jede Gruppe beim erstellen eine eindeutiges Token(UUID, durch die sie eindeutig identifziert werden kann) erhält, haben wir das Token der Testgruppe mit dem Token der Gruppe, die addGroup zurückgegeben hat, verglichen.		
		}
		
		\item {Gruppe aus dem Repository holen: \\
		Erster Test: Eine Testgruppe wurde mit der Methode Group addGroup(Group group)dem Repository hinzugefügt. Mit der Methode Group getGroup(Token token) wurde die eingefügte Gruppe aus dem Repository geholt. Es wurde überprüft, dass das Token der geholten Gruppe wirklich dem Token der Testgruppe entspricht und somit das holen der Gruppe korrekt funktioniert. \\
		\\ Zweiter Test: Die Testgruppe wurde nicht ins Repository eingefügt. Die Methode Group getGroup(Token token) wurde mit dem Token der Testgruppe aufgerufen. Da die Gruppe nicht im Repository ist, wird wie erwartet eine ResponseStatusException geworfen. }
		
		\item {(Alle) Gruppe(n) aus dem Repository entfernen: \\
		Erster Test: Eine Testgruppe wurde mit der Methode Group addGroup(Group group)dem Repository hinzugefügt und mit removeGroup(String groupToken) wieder aus dem Repository entfernt. Danach wurde Group[] getAllGroups() aufgerufen und überprüft ob das zurückgegebene Array die Länge null hat. \\
		\\ Zweiter Test: Zwei Testgruppen wurden jeweils mit der Methode Group addGroup(Group group)dem Repository hinzugefügt. Danach wurde die Methode removeAllGroups(), die alle Gruppen aus dem Repository löschen soll. Das wurde überprüft indem geprüft wurde, ob die Länge des Arrays, das von der Methode Group[] getAllGroups() zurückgegeben wird, gleich 0 ist.
		}
		
		\item {Nicht bestehende Gruppe aus dem Repository holen: \\
		
		}
		
	\end{itemize}
\end{itemize}


\subsection{Client}


\printglossaries
\end{document}
