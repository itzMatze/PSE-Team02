\documentclass[a4paper]{scrreprt}
\setcounter{tocdepth}{3}
\setcounter{secnumdepth}{3}

\usepackage[german]{babel}
\usepackage[utf8]{inputenc}
\usepackage[T1]{fontenc}
\usepackage{ae}
\usepackage{graphicx}
\usepackage{lscape} % querformat
\usepackage{tabu}
\usepackage{hyperref}
\usepackage{xcolor}
\usepackage[toc]{glossaries}

\makeglossaries

\newglossaryentry{Unittest}
{
  name=Unittest,
  description={Auch bekannt als Modul- oder Komponententest. Wird in der Softwareentwicklung angewendet, um die funktionalen Einzelteile (Units) von Computerprogrammen zu testen, d. h., sie auf korrekte Funktionalität zu prüfen}
}

\newglossaryentry{Hallway Usability Test}
{
  name=Hallway Usability Test,
  description={Test, bei dem zufällige Personen zum Testen von Softwareprodukten und -schnittstellen verwendet werden}
}

\newglossaryentry{Qualitaetssicherung}
{
  name=Qualitätssicherung,
  description={Letzte Phase des Wasserfallmodells}
}


\begin{document}
\title{Testbericht}
\author{Fangzhou Bian, Kathrin Blum, Matthias Bruns, \\Leonhard Duda, Tan Grumser, Yuguang Lin}
\maketitle
%\Footnote für Fußnoten
% Platzierung des Inhaltsverzeichnisses
\tableofcontents



\chapter{Einleitung}

Dieses Dokument verschafft einen Überblick über die \Gls{Qualitaetssicherung} des Projekts. In dieser Phase wurde die Überdeckung der \Gls{Unittest}s maximiert, die Testszenarien des Pflichtenhefts durchgegangen und gefundene Fehler behoben. Des Weiteren wurde ein \Gls{Hallway Usability Test} durchgeführt. Die dokumentierten Tests wurden wieder unterteilt in einen Server- und Clientteil.

\chapter{Unittests}
\section{Server}
\section{Client}

\chapter{Testszenarien}

\chapter{Hallway Usability Testing}

\printglossaries
\end{document}