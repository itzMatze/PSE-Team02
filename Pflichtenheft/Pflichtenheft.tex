\documentclass[a4paper]{scrreprt}
 
\usepackage[german]{babel}
\usepackage[utf8]{inputenc}
\usepackage[T1]{fontenc}
\usepackage{ae}
\usepackage[bookmarks,bookmarksnumbered]{hyperref}
 
\begin{document}
 
\title{Pflichtenheft für MensaMeat}

\maketitle
%\Footnote für Fußnoten
 
% Platzierung des Inhaltsverzeichnisses
\tableofcontents
 
\chapter{Zielbestimmung}
Mit dieser App sollen Personen(Studenten), die in der Mensa essen möchten, in die Lage versetzt werden, sich mit ihnen unbekannten Personen (Studenten) zum Essen verabreden zu können. Auf diese Art können leicht neue Menschen kennengelernt werden und keiner muss nicht alleine essen.
 
\section{Musskriterien}
\begin{itemize}


\item Erstellen und Bearbeiten eines persönlichen Profils (Nutzername(eindeutig), Alter, Geschlecht, Studiengang, Motto (Freitextfeld), Sprachen)
\item Einsehen des Mensaplans des aktuellen Tages
\item Auswählen der Mensalinien/Mensawerke
\item Einsehen der Gruppen, mit übereinstimmenden Mensalnien/Werken
\item Erstellen einer Gruppe
\item Beitreten einer Gruppe
\item Verlassen einer Gruppe
\item Anschauen der Profile der Gruppenmitglieder
\item Automatisches Löschen der Gruppe am Ende des Tages, bzw wenn letzes Mitglied austritt
\end{itemize}

\section{Wunschkriterien}
\begin{itemize}

\item Erweiterte Profileinstellungen (Mag, Mag nicht, Vegetarier/Veganer,... )
\item E-Mail Verifizierung für die Erstellung eines Profils 
\item Bitmoji erstellen, welches als Profilbild angezeigt wird
\item (gern) Ranking System um die Zuverlässigkeit einer Person einzuschätzen. Das Erscheinen zu den Treffen bringt Punkte, von denen der Rang abhängt
\item Gruppenmitglieder können angeben, welche Personen zur Verabredung erschienen bzw. nicht erschienen sind, dies wirkt sich auf das Ranking aus
\item Einstellbare Restriktionen für den Gruppenbeitritt, z.b. nach Geschlecht, Studiengang, Veganismus, Rang, Alter
\item Filtermöglichkeiten der angezeigten Gruppen z.b. nach Treffzeitpunkt, aktuelle und maximale Mitgliederanzahl 
\item Einsicht in den Mensaplan der folgenden Tage und
\item Erstellen einer Gruppe für folgende Tage
\item Melden von Personen ("Snapshot" der Sicht aus der gemeldet wird, zum beurteilen an Administratoren gesandt) 
\end{itemize}
 
\section{Abgrenzungskriterien}
\begin{itemize}

\item Kein Upload von (Profil-)Bildern möglich
\item Kein "Kicken" von Personen aus erstellter Gruppe
\item Kein Chatsystem
\item Kein Löschen von Gruppen (denn Gruppengründer hat gleichwertige Rolle zu anderen Mitgliedern, er kann selbst austreten.)

\end{itemize}
 
\chapter{Produkteinsatz}
Das Produkt wird von Personen(Studenten) eingesetzt, um sich mit unbekannte Personen (Studenten) zum gemeinsamen Essen in der Mensa zu verabreden.

 
\section{Anwendungsbereiche}
Essensplanung an der Mensa (Wann, mit Wem, an welche Linie).
 
\section{Zielgruppen}
Mensagänger; Studenten, Professoren, Auszubildende 
 
\section{Betriebsbedingungen}
\subsection{Physikalische Umgebung}
Zuhause, Campus, Unterwegs(in der Stadt, in der Straßenbahn)

\subsection{Tägliche Betriebszeit}
erwartet bis zu 3 mal täglich, jeweils weniger als 10 Minuten am Stück 
 
\chapter{Produktumgebung}
Das Produkt läuft als App auf einem Smartphone

\section{Software}
Andorid, ab Version 4.4?
 
\section{Hardware}
Smartphone 
 
\section{Orgware}

\section{Produktschnittstellen} 
 
\chapter{Funktionale Anforderungen}
TODO Sie werden eingeteilt in Client und Server Funktionalitäten.

\section{Client}
Auf dem Client (in diesem Fall ein Smatphone) müssen alle folgende Anforderungen erfüllt sein. 

\subsection{Allgemein}

\begin{addmargin}[25pt]{0pt} 
\textbf{/CAFA10/} Installation der Applikation \\
\textbf{/CAFA20/} Starten / Beenden der Applikation\\
\textbf{/CAFA30/} Speciherung lokaler Daten (Login Daten)\\
\end{addmargin}

\subsection{User Interface}

\begin{addmargin}[25pt]{0pt} 
\textbf{/CIFA10/} Anmelden / Registrieren \\
\textbf{/CIFA20/} Angaben persönlicher Daten\\
\textbf{/CIFA30/} Anzeigen und Auswählen der Essenslinien\\
\textbf{/CIFA40/} Angeben des bevorzugten Zeitraums\\
\textbf{/CIFA50/} Anzeigen und Auswählen einer Gruppe\\
\textbf{/CIFA60/} Anzeigen eines Userprofils\\
Anzeigen des Regelwerks\\
Anzeigen der Gruppenseite\\
Einstellngen des Profils\\
\end{addmargin}

\section{Server}

\begin{addmargin}[25pt]{0pt} 
\textbf{/SFA10/} User anlegen / löschen\\
\textbf{/SFA20/} User anmelden / abmelden\\
\textbf{/SFA30/} Gruppe erstellen mit Parametern\\
\textbf{/SFA40/} Gruppe löschen\\
	\begin{addmargin}[25pt]{0pt} 
	\textbf{/SFA41/} Durch Autretten des letzten Users\\
	\textbf{/SFA41/} Durch Admin\\
	\textbf{/SFA42/} Automatisch durch Timeout\\
	\end{addmargin}
\textbf{/SFA41/} Anzeigen der Gruppen zu Parametern\\
\textbf{/SFA120/} User tritt Gruppe bei\\
\textbf{/SFA130/} User verlässt Gruppe\\
\textbf{/SFA130/} Informationen bearbeiten von User\\
\textbf{/SFA120/} Abfragen der Mensapläne\\
\textbf{/SFA41/} Abfragen von Userprofil\\
\end{addmargin}

\chapter{Produktdaten}
\begin{itemize}
\item /D10/ 
\item /D20A/
%nach dem selben Schema weiter; \item und was dahinter stehen soll %
\end{itemize}

\chapter{Nichtfunktionale Anforerungen}

\begin{addmargin}[25pt]{0pt} 
\textbf{/NFA10/} Antworten vom Server dürfen nicht später als eine Sekunde nach Anfrage beim Client eingehen.\\
\textbf{/NFA20/} Es können bis zu 10.000 User in der Datenbank angelegt werden.\\
\textbf{/NFA30/} Eingaben dürfen nicht länger als 100 Zeiche sein.\\
\textbf{/NFA40/} Es können beim Suchen passender Gruppen bis zu 30 Ergebnisse geliefert werden.\\
\textbf{/NFA50/} Änderung von Informationen sollen in unter 100ms systemweit bekannt sein.\\
\textbf{/NFA60/} Das System muss parallel von bis zu 1000 Usern benutzt werden können.\\
\textbf{/NFA70/} Das System darf nicht mehr als einen Neustart pro Woche brauchen.\\
\end{addmargin}


\chapter{Gloale Testfälle}
\begin{itemize}
\item /T10/ 
\item /T20/

\end{itemize}

\chapter{Systemmodelle}

\chapter{Glossar}
 

 
\end{document}